\chapter{Introduzione}
\label{chap:chap1}

La teoria degli \emph{insiemi fuzzy}, introdotta da Zadeh negli anni ’60~\cite{Goguen_1973}, ha rappresentato una svolta nella modellazione dell’incertezza e dell’imprecisione nei sistemi informatici. La sua caratteristica distintiva risiede nella possibilità di esprimere l’appartenenza graduale di un elemento a un insieme, superando la rigidità binaria degli insiemi classici. Questo approccio ha trovato applicazioni in una vasta gamma di domini, tra cui il controllo automatico, l’analisi semantica, il data mining e, più recentemente, i \emph{sistemi di raccomandazione}.

I \emph{sistemi di raccomandazione} sono oggi componenti centrali delle piattaforme digitali, come Netflix, Amazon o Spotify, dove svolgono il compito di suggerire contenuti rilevanti agli utenti, basandosi sulle loro preferenze esplicite o implicite~\cite{554776b9726f44c582f01d870cefd26b}. Tuttavia, la natura ambigua e incerta delle preferenze umane rende complesso modellarle con tecniche tradizionali. In tale contesto, la logica fuzzy offre strumenti promettenti per rappresentare tali incertezze e preferenze in modo più flessibile e realistico~\cite{EKEL2006179}.

L'\emph{Obiettivo} del seguente approfondimento è esplorare l’utilizzo del \textbf{fuzzy clustering} nei sistemi di raccomandazione, con particolare attenzione all’ambito della raccomandazione personalizzata di contenuti audiovisivi. In particolare, si propone un’estensione metodologica di quanto svolto in~\cite{KOOHI2016134}, dove si applica l’algoritmo \emph{Fuzzy C-Means (FCM)} per raggruppare gli utenti in cluster sovrapposti, modellando così la possibilità che un utente appartenga simultaneamente a più gruppi di preferenza.

A partire da questo contributo, il progetto realizzato si propone di esplorare l’impatto di diverse tecniche di normalizzazione, parametri di clustering e rumore sulle performance del sistema, fornendo un framework riproducibile per valutare il clustering fuzzy nel filtraggio collaborativo tramite metriche quantitative e analisi visive.

La seguente relazione è strutturata come segue:

\begin{itemize}
    \item \textbf{\hyperref[chap:chap2]{Capitolo~\ref*{chap:chap2}}} presenta i fondamenti teorici relativi alla logica fuzzy, al fuzzy clustering e ai sistemi di raccomandazione;
    \item \textbf{\hyperref[chap:chap3]{Capitolo~\ref*{chap:chap3}}} descrive l'analisi esplorativa del dataset MovieLens 100k;
    \item \textbf{\hyperref[chap:chap4]{Capitolo~\ref*{chap:chap4}}} descrive l'implementazione del sistema di raccomandazione fuzzy;
    \item \textbf{\hyperref[chap:chap5]{Capitolo~\ref*{chap:chap5}}} presenta i risultati degli esperimenti condotti;
    \item \textbf{\hyperref[chap:chap6]{Capitolo~\ref*{chap:chap6}}} conclude il lavoro e propone linee di ricerca future.
\end{itemize}
